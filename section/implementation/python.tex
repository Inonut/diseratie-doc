\subsection{Why python?}

Python is an interpreted, object-oriented, high-level programming language with dynamic semantics. Its high-level built in data structures, combined with dynamic typing and dynamic binding, make it very attractive for Rapid Application Development, as well as for use as a scripting or glue language to connect existing components together. Python's simple, easy to learn syntax emphasizes readability and therefore reduces the cost of program maintenance. Python supports modules and packages, which encourages program modularity and code reuse. The Python interpreter and the extensive standard library are available in source or binary form without charge for all major platforms, and can be freely distributed.

Often, programmers fall in love with Python because of the increased productivity it provides. Since there is no compilation step, the edit-test-debug cycle is incredibly fast. Debugging Python programs is easy: a bug or bad input will never cause a segmentation fault. Instead, when the interpreter discovers an error, it raises an exception. When the program doesn't catch the exception, the interpreter prints a stack trace. A source level debugger allows inspection of local and global variables, evaluation of arbitrary expressions, setting breakpoints, stepping through the code a line at a time, and so on. The debugger is written in Python itself, testifying to Python's introspective power. On the other hand, often to add a few print statements to the source is the quickest way to debug a program: the fast edit-test-debug cycle makes this simple approach very effective.

Python language is one of the most flexible languages. It can be used for various purposes. Python has gained huge popularity base of this. Python does contain special libraries for machine learning namely scipy and numpy which great for linear algebra and getting to know kernel methods of machine learning. When want to work with machine learning algorithms python is great to use. It has easy syntax relatively. For beginners, this is the best language to use and to start with.

Whether a startup or an MNC, Python provides a huge list of benefits to all. The usage of Python is such that it cannot be limited to only one activity. Its growing popularity has allowed it to enter into some of the most popular and complex processes like Artificial Intelligence (AI), Machine Learning (ML), natural language processing, data science etc. The question is why Python is gaining such momentum in AI? And the answer lies below:

\begin{itemize}
    \item Less Code: AI involves algorithms - a LOT of them. Python provides ease of testing -  one of the best among competitors. Python helps in easy writing and execution of codes. Python can implement the same logic with as much as 1/5th code as compared to other OOPs languages. Thanks to its interpreted approach which enables check as you code methodology.
    \item Prebuilt Libraries: Python has a lot of libraries for every need of your AI project. Few names include Scipy for advanced computing, Numpy for scientific computation and Pybrain for machine learning. AIMA - is a library implemented in Python of algorithms from Russell and Norvig's 'Artificial Intelligence: A Modern Approach' is one of the best library available for Artificial Intelligence till today. Such a dedicated library saves developer’s time spent on coding base level items.
    \item Support: Python is a completely open source with a great community. There is a host of resources available which can get any developer up to speed in no time. There is a huge community of active coders willing to help programmers in every stage of developing cycle.
    \item Platform Agnostic: Python provides the flexibility to provide an API from an existing language which indeed provides extreme flexibility. It is also platform independent. With just a few changes in codes, you can get your app up and running in a new OS. This saves developers time in testing on different platforms and migrating code.
    \item Flexibility: Flexibility is one of the core advantages of Python. Python is suitable for every purpose with the option to choose between OOPs approach and scripting. It works as a perfect backend and it also suitable for linking different data structures together. A big plus for developers who are struggling between different algorithms is the option to check a majority of code in the IDE itself.
    \item Popularity: Python is winning the heart of millennials. It's ease of learning. Though AI Projects need a highly experienced programmer yet Python can smoothen the learning curve. It is practically more easy to look for Python developers than to hunt for LISP or Prolog programmers, particularly in some nations. Its extended libraries and active community with an ever developing and improving code have led it to be one of the hottest languages today.
\end{itemize}

Python is simple, elegant, consistent, and math-like.