\section*{Conclusion}
\addcontentsline{toc}{section}{Conclusion}

The work done represents a special incursion in the fundamental field of neural networks, both by presenting the main elements that define this field and by the illustrative character of the experimental experimentation of the neural algorithms performance in applications of great interest in the current researches.

In the paper were presented the notions defining the field of neural networks, by describing the neuron structure, highlighting the comparison between the artificial and the human neuron, the possible training possibilities for the unilateral or multilayer neuronal architectures by using information about the correct answers associated with the data training and so we have supervised learning, by determining as far as possible the final clusters that induce unsupervised learning, as well as enumerating the main areas of applicability. It is possible to specify the numerous applications present in the field of form recognition with the use of neural network-specific training algorithms.

The tensorflow framework was used in the elaboration of the paper. It provides standard deployments for a wide range of optimizations for neural networks, architecture, and more. We also used the uncontrolled neural network GloVe to train words and find correspondence between them. These tools have been applied to find and classify names in texts.

