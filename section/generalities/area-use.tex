\subsection{Areas of use of neural networks}

\subsubsection{Processing of natural languages}

Understanding natural languages allows people to interact with computers without special knowledge.

D. Rumelhart and J. McClelland \cite{Rumelhart} introduced neural networks in the field of natural language processing. By processing a natural language we will understand the study of how to construct the rules of a language.

D. Rumelhart and J. McClelland studied a language learning model with the help of a neural network able to teach English past Past Tense. By learning the neural network, it has progressed from a beginner's stage that brings bring-broughted mistakes to a phase of specialist in which she was able to determine the time for irregular verbs. The ability of the neural network to generalize on the basis of incomplete data and self-organization has allowed the neural network to generate correct responses when a new or unknown verb was presented.

\subsubsection{Compression of multidimensional data}

G. W. Cottrell, D. Zipser and P. Munro \cite{Cottrell} used neural networks to efficiently compress information corresponding to graphical images. The graphical images occupy, according to the resolution of the representation and the number of colors used, a very large storage space, reaching the order of the mega-bytes.

Image compression is a practical necessity because the storage space is very expensive and at the same time the transfer time of an image is obviously influenced by the amount of memory space required for that image.

The neural computer system designed by Cottrell, Munro, and Zipser is based on a three-layer neural network capable of compressing an image, and of course able to decompress it without distortion. It is important to note the unsupervised learning law used to learn the neural network, which allowed it to self-configure without the intervention of specialists. With this neural network, data compression has been accomplished eight times, with an impeccable decompression of the original image.


\subsubsection{Character Recognition}

An important area of use of neural networks is the field of visual interpretation and symbol classification \cite{Bishop}.

\paragraph{Handwriting recognition.} Researchers at Nestor Inc. in the US, have developed a neural computer system that has as a data entry device a digitizing tablet, which can be written using a Light-Pen. The neural network was trained with different handwriting, being able to interpret some handwriting with a high degree of acuity.

There are a large number of optical character recognition systems called OCR (Optical Character Recognition) \cite{oecd}. What differentiates neural networks from traditional OCR systems is flexibility. After learning, the neural network is able to recognize a great variety of writings and make pertinent assumptions about confusing characters.

Nestor's researchers built a neural network for Japanese writing (Kanji). It has been possible to eliminate the difficulty of quantifying the specific elements of a language by using neural networks in this area.

\paragraph{Image processing.} K. Fukushima \cite{Fukushima1,Fukushima2} has developed a neural imaging system for image recognition with practical applicability in the field of character recognition. The built-in neural network is based on an advanced form recognition system called Neocognitron.

Neocognitron is actually a multi-layer neural network. That simulates the way the human cortex is processed. The successively hidden neurons of Neocognitron are designed to extract defining features of the image without being influenced by orientation or distortion. At the input layer level, shapes are uniquely determined, with the information being propagated to the output layer, only certain neurons corresponding to some defining features of the image being activated.

